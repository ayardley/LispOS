\documentclass{slides}
\usepackage[utf8]{inputenc}
\usepackage{graphics}
\usepackage{portland}
\usepackage{epsfig}
\usepackage{alltt}
\usepackage{moreverb}
\usepackage{url}
\usepackage[dvips,usenames]{color}

\definecolor{MyLightMagenta}{rgb}{1,0.7,1}
\definecolor{darkgreen}{rgb}{0.1,0.7,0.1}

\newcommand{\darkgreen}[1]{\textcolor{darkgreen}{#1}}
\newcommand{\red}[1]{\textcolor{red}{#1}}
\newcommand{\thistle}[1]{\textcolor{Thistle}{#1}}
\newcommand{\apricot}[1]{\textcolor{Apricot}{#1}}
\newcommand{\melon}[1]{\textcolor{Melon}{#1}}
\newcommand{\dandelion}[1]{\textcolor{Dandelion}{#1}}
\newcommand{\green}[1]{\textcolor{OliveGreen}{#1}}
\newcommand{\lavender}[1]{\textcolor{Lavender}{#1}}
\newcommand{\mylightmagenta}[1]{\textcolor{MyLightMagenta}{#1}}
\newcommand{\blue}[1]{\textcolor{RoyalBlue}{#1}}
\newcommand{\darkorchid}[1]{\textcolor{DarkOrchid}{#1}}
\newcommand{\orchid}[1]{\textcolor{Orchid}{#1}}
\newcommand{\brickred}[1]{\textcolor{BrickRed}{#1}}
\newcommand{\peach}[1]{\textcolor{Peach}{#1}}
\newcommand{\bittersweet}[1]{\textcolor{Bittersweet}{#1}}
\newcommand{\salmon}[1]{\textcolor{Salmon}{#1}}
\newcommand{\yelloworange}[1]{\textcolor{YellowOrange}{#1}}
\newcommand{\periwinkle}[1]{\textcolor{Periwinkle}{#1}}

\newcommand{\names}[1]{\periwinkle{#1}}
\newcommand{\motcle}[1]{\mylightmagenta{#1}}
\newcommand{\classname}[1]{\darkgreen{#1}}
\newcommand{\str}[1]{\yelloworange{#1}}
\newcommand{\defun}[1]{\orchid{#1}}
\newcommand{\ti}[1]{\begin{center}\Large{\textcolor{blue}{#1}}\end{center}}
\newcommand{\alert}[1]{\thistle{#1}}
\newcommand{\lispprint}[1]{\dandelion{#1}}
\newcommand{\lispvalue}[1]{\red{#1}}
\newcommand{\tr}[1]{\texttt{\red{#1}}}
\newcommand{\emc}[1]{\red{#1}}
\newcommand{\lispobj}[1]{\green{\texttt{#1}}}
\def\prompt{{\textcolor{Orchid}{CL-USER>}}}
\newcommand{\promptp}[1]{\textcolor{Orchid}{#1>}}

\newcommand{\Comment}[1]{
\begin{center}
\textcolor{yellow}
{#1}
\end{center}
}

\def\bs{$\backslash$}
\def\inputfig#1{\input #1}
\def\inputtex#1{\input #1}

\begin{document}
\landscape
\setlength{\oddsidemargin}{1cm}
\setlength{\evensidemargin}{1cm}
\setlength{\marginparwidth}{1cm}
\setlength{\parskip}{0.5cm}
\setlength{\parindent}{0cm}
%-----------------------------------------------------------
\begin{slide}\ti{My ideal operating system}
\vskip 0.5cm
\begin{center}
Robert Strandh \\
LaBRI \\
Université de Bordeaux \\
Bordeaux, France
\end{center}
\vfill\end{slide}
%-----------------------------------------------------------
\begin{slide}\ti{Problems with existing systems}
\vskip 0.5cm
\begin{itemize}
\item The \emph{process} concept
\item Different semantics of primary and secondary memory
\item Hierarchical file systems. 
\item The existence of a \emph{kernel}
\item Monolithic applications
\end{itemize}
\vfill\end{slide}
%-----------------------------------------------------------
\begin{slide}\ti{The process concept}
\vskip 0.5cm
Recall what a process is:

Each application gets the illusion of having the full physical address
space of the physical processor.  

Problems:
\begin{itemize}
\item An address (pointer) can not be transmitted from one process to
  another process.
\item Complex data structures must be serialized and de-serialized in
  order to be communicated.
\end{itemize}
\vfill\end{slide}
%-----------------------------------------------------------
\begin{slide}\ti{Hierarchical file systems}
\vskip 0.5cm
Problematic in two ways:

\begin{itemize}
\item Hierarchical
\item File
\end{itemize}

\vfill\end{slide}
%-----------------------------------------------------------
\begin{slide}\ti{Problem with hierarchies}
\vskip 0.5cm
Information is really not hierarchical. 

\texttt{.../Articles/Lisp/2015/...}\\
\texttt{.../Articles/2015/Lisp/...}\\
\texttt{.../Lisp/Articles/2015/...}\\
\texttt{.../Lisp/2015/Articles/...}\\
\texttt{.../2015/Lisp/Articles/...}\\
\texttt{.../2015/Articles/Lisp/...}\\

Are typically all the same.  How to organize?

\vfill\end{slide}
%-----------------------------------------------------------
\begin{slide}\ti{Problem with files}
\vskip 0.5cm
In most modern systems a \emph{file} is a sequence of 8-bit bytes.

But we would like to store other things too:

\begin{itemize}
\item People
\item Organizations
\item etc.
\end{itemize}

\vfill\end{slide}
%-----------------------------------------------------------
\begin{slide}\ti{The process concept}

When Unix was invented, the machine on which it ran had no
memory-management unit (MMU).  The solution was to create every
program so that it was located at address 0, and to \emph{swap}.

\vfill\end{slide}
%-----------------------------------------------------------
\begin{slide}\ti{Hierarchies}

Maybe Multics 1965.

\vfill\end{slide}
%-----------------------------------------------------------
\begin{slide}\ti{Files as sequences of bytes}

Multics had no files.  Instead it had \emph{segments} which were
arrays of bytes.

Unix defined a file as a sequence of bytes.

\vfill\end{slide}
%-----------------------------------------------------------
\begin{slide}\ti{The existence of a kernel}

Multics did not have a kernel.

Unix did, but the concept is probably older.

\vfill\end{slide}
%-----------------------------------------------------------
\begin{slide}\ti{Solution}

Let's examine what the solution might be.

\vfill\end{slide}
%-----------------------------------------------------------
\begin{slide}\ti{Process concept}

A single address space.

Applications must not have access.

Possible now because of 64-bit addresses.

\vfill\end{slide}
%-----------------------------------------------------------
\begin{slide}\ti{}

\vfill\end{slide}
%-----------------------------------------------------------
\begin{slide}\ti{}

\vfill\end{slide}
%-----------------------------------------------------------
\begin{slide}\ti{Technology that helps}

  \begin{itemize}
  \item Large address space.
  \item New permanent memory (3D Xpoint).
  \item Multi-core processors making concurrent garbage collection
    possible. 
  \item New concurrent garbage-collection algorithms.
  \end{itemize}

\vfill\end{slide}
%--------------------------------
\begin{slide}\ti{Current status}
\vfill\end{slide}
%-----------------------------------------------------------
\begin{slide}\ti{Conclusions}

\vfill\end{slide}
%-----------------------------------------------------------
\begin{slide}\ti{Future work}

\vfill\end{slide}
%% %-----------------------------------------------------------
%% \begin{slide}\ti{}

%% \vfill\end{slide}
%% %-----------------------------------------------------------
%% \begin{slide}\ti{}

%% \vfill\end{slide}
%% %-----------------------------------------------------------
%% \begin{slide}\ti{}

%% \vfill\end{slide}
%--------------------------------

\end{document}
 
