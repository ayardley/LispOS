\documentclass{slides}
\usepackage[utf8]{inputenc}
\usepackage{graphics}
\usepackage{portland}
\usepackage{epsfig}
\usepackage{alltt}
\usepackage{moreverb}
\usepackage{url}
\usepackage[dvips,usenames]{color}

\definecolor{MyLightMagenta}{rgb}{1,0.7,1}
\definecolor{darkgreen}{rgb}{0.1,0.7,0.1}

\newcommand{\darkgreen}[1]{\textcolor{darkgreen}{#1}}
\newcommand{\red}[1]{\textcolor{red}{#1}}
\newcommand{\thistle}[1]{\textcolor{Thistle}{#1}}
\newcommand{\apricot}[1]{\textcolor{Apricot}{#1}}
\newcommand{\melon}[1]{\textcolor{Melon}{#1}}
\newcommand{\dandelion}[1]{\textcolor{Dandelion}{#1}}
\newcommand{\green}[1]{\textcolor{OliveGreen}{#1}}
\newcommand{\lavender}[1]{\textcolor{Lavender}{#1}}
\newcommand{\mylightmagenta}[1]{\textcolor{MyLightMagenta}{#1}}
\newcommand{\blue}[1]{\textcolor{RoyalBlue}{#1}}
\newcommand{\darkorchid}[1]{\textcolor{DarkOrchid}{#1}}
\newcommand{\orchid}[1]{\textcolor{Orchid}{#1}}
\newcommand{\brickred}[1]{\textcolor{BrickRed}{#1}}
\newcommand{\peach}[1]{\textcolor{Peach}{#1}}
\newcommand{\bittersweet}[1]{\textcolor{Bittersweet}{#1}}
\newcommand{\salmon}[1]{\textcolor{Salmon}{#1}}
\newcommand{\yelloworange}[1]{\textcolor{YellowOrange}{#1}}
\newcommand{\periwinkle}[1]{\textcolor{Periwinkle}{#1}}

\newcommand{\names}[1]{\periwinkle{#1}}
\newcommand{\motcle}[1]{\mylightmagenta{#1}}
\newcommand{\classname}[1]{\darkgreen{#1}}
\newcommand{\str}[1]{\yelloworange{#1}}
\newcommand{\defun}[1]{\orchid{#1}}
\newcommand{\ti}[1]{\begin{center}\Large{\textcolor{blue}{#1}}\end{center}}
\newcommand{\alert}[1]{\thistle{#1}}
\newcommand{\lispprint}[1]{\dandelion{#1}}
\newcommand{\lispvalue}[1]{\red{#1}}
\newcommand{\tr}[1]{\texttt{\red{#1}}}
\newcommand{\emc}[1]{\red{#1}}
\newcommand{\lispobj}[1]{\green{\texttt{#1}}}
\def\prompt{{\textcolor{Orchid}{CL-USER>}}}
\newcommand{\promptp}[1]{\textcolor{Orchid}{#1>}}

\newcommand{\Comment}[1]{
\begin{center}
\textcolor{yellow}
{#1}
\end{center}
}

\def\bs{$\backslash$}
\def\inputfig#1{\input #1}
\def\inputtex#1{\input #1}

\begin{document}
\landscape
\setlength{\oddsidemargin}{1cm}
\setlength{\evensidemargin}{1cm}
\setlength{\marginparwidth}{1cm}
\setlength{\parskip}{0.5cm}
\setlength{\parindent}{0cm}
%-----------------------------------------------------------
\begin{slide}\ti{My ideal operating system}
\vskip 0.5cm
\begin{center}
Robert Strandh \\
LaBRI \\
Université de Bordeaux \\
Bordeaux, France
\end{center}
\vfill\end{slide}
%-----------------------------------------------------------
\begin{slide}\ti{Overview of talk}
\vskip 0.5cm
  \begin{itemize}
  \item Problems with existing systems.
  \item Why we are in this fix.
  \item Proposed solutions.
  \end{itemize}

\vfill\end{slide}
%-----------------------------------------------------------
\begin{slide}\ti{Questions I ask myself}
\vskip 0.5cm
As a user:

  \begin{itemize}
  \item Why do I have to click on a \emph{save} button?
  \item Why do I have to restart an application after an update?
  \item Why do I have to restart the computer after kernel updates?
  \item Why do I have to remember the order between directories in a
    file name?
  \end{itemize}

\vfill\end{slide}
%-----------------------------------------------------------
\begin{slide}\ti{Questions I ask myself}
\vskip 0.5cm
As a programmer:

  \begin{itemize}
  \item Why do I have to serialize my data structure or turn it into
    something acceptable to SQL in order to avoid losing it?
  \item Why do I have to serialize my data structure to communicate it
    to a different application?
  \item Why do we have address-space randomization?
  \item Why is my computer vulnerable to viruses, worms, and other
    attacks?
  \item Why do we need all these file formats (PDF, PNG, JPEG, MP3,
    MPEG, etc)?
  \end{itemize}

\vfill\end{slide}
%-----------------------------------------------------------
\begin{slide}\ti{Questions I ask myself}
\vskip 0.5cm
As a teacher:

  \begin{itemize}
  \item Why do we teach outdated technology, even to students at the
    masters level?
  \item Why do text books on operating-system technology not criticize
    current technology?
  \end{itemize}

\vfill\end{slide}
%-----------------------------------------------------------
\begin{slide}\ti{Questions I ask myself}
\vskip 0.5cm
As a researcher:

  \begin{itemize}
  \item Why is there no funding for research in operating systems?
  \end{itemize}

\vfill\end{slide}
%-----------------------------------------------------------
\begin{slide}\ti{Problems with existing systems}
\vskip 0.5cm
\begin{itemize}
\item The \emph{process} concept
\item Different semantics of primary and secondary memory
\item Hierarchical file systems. 
\item The existence of a \emph{kernel}
\item Monolithic applications
\end{itemize}
\vfill\end{slide}
%-----------------------------------------------------------
\begin{slide}\ti{The process concept}
\vskip 0.5cm
Recall what a process is:

Each application gets the illusion of having the full physical address
space of the physical processor.  

Problems:
\begin{itemize}
\item An address (pointer) can not be transmitted from one process to
  another process.
\item Complex data structures must be serialized and de-serialized in
  order to be communicated.
\end{itemize}
\vfill\end{slide}
%-----------------------------------------------------------
\begin{slide}\ti{Hierarchical file systems}
\vskip 0.5cm
Problematic in two ways:

\begin{itemize}
\item Hierarchical
\item File
\end{itemize}

\vfill\end{slide}
%-----------------------------------------------------------
\begin{slide}\ti{Problem with hierarchies}
\vskip 0.5cm
Information is really not hierarchical. 

\texttt{.../Articles/Lisp/2015/...}\\
\texttt{.../Articles/2015/Lisp/...}\\
\texttt{.../Lisp/Articles/2015/...}\\
\texttt{.../Lisp/2015/Articles/...}\\
\texttt{.../2015/Lisp/Articles/...}\\
\texttt{.../2015/Articles/Lisp/...}\\

Are typically all the same.  How to organize?

Google for ``Ontology overrated''.

\vfill\end{slide}
%-----------------------------------------------------------
\begin{slide}\ti{Problem with kernel}
\vskip 0.5cm
\begin{itemize}
\item If the kernel is updated, the computer must be restarted.
\item Multics (1964) did not have a kernel.  System code could be
  dynamically replaced.
\end{itemize}

\vfill\end{slide}
%-----------------------------------------------------------
\begin{slide}\ti{Problem with files}
\vskip 0.5cm
In most modern systems a \emph{file} is a sequence of 8-bit bytes.

But we would like to store other things too:

\begin{itemize}
\item People
\item Organizations
\item etc.
\end{itemize}

\vfill\end{slide}
%-----------------------------------------------------------
\begin{slide}\ti{Problem with monolithic applications}

  \begin{itemize}
  \item If an application is updated, it must be restarted.
  \item An application has full access to its address space, including
    the stack, library code, etc.
  \end{itemize}

\vfill\end{slide}
%-----------------------------------------------------------
\begin{slide}\ti{Origin of problems}

What is the origin of those problems?

\vfill\end{slide}
%-----------------------------------------------------------
\begin{slide}\ti{Different memory semantics}

Early computers had primary memory made of magnetic cores and
secondary memory in the form of sequential tape.

(not however, that core memory was non-volatile)

Initially, files were created to behave like tape.

Multics (1964) unified primary and secondary memory.

\vfill\end{slide}
%-----------------------------------------------------------
\begin{slide}\ti{The process concept}

When Unix was invented, the machine on which it ran had no
memory-management unit (MMU).

The solution was to create every program so that it was located at
address 0, and to \emph{swap}.

\vfill\end{slide}
%-----------------------------------------------------------
\begin{slide}\ti{Hierarchies}

Maybe Multics 1965.

\vfill\end{slide}
%-----------------------------------------------------------
\begin{slide}\ti{Files as sequences of bytes}

Multics had no files.  Instead it had \emph{segments} which were
arrays of bytes.

Unix defined a file as a sequence of bytes.

\vfill\end{slide}
%-----------------------------------------------------------
\begin{slide}\ti{The existence of a kernel}

Multics did not have a kernel.

Unix did, but the concept is probably older.

\vfill\end{slide}
%-----------------------------------------------------------
\begin{slide}\ti{Monolithic applications}

  \begin{itemize}
  \item The desire to use linker technology from the 1950s.
  \item But Multics (1964) had a dynamic linker.
  \end{itemize}

\vfill\end{slide}
%-----------------------------------------------------------
\begin{slide}\ti{Solution}

Let's examine what the solution might be.

\vfill\end{slide}
%-----------------------------------------------------------
\begin{slide}\ti{Process concept}

A single address space.

Applications must not have access.

Possible now because of 64-bit addresses.

\vfill\end{slide}
%-----------------------------------------------------------
\begin{slide}\ti{Primary and secondary memory}

Orthogonal persistence.

\vfill\end{slide}
%-----------------------------------------------------------
\begin{slide}\ti{Hierarchy}

Object store based on tags.

Tags are key/value pairs.

\vfill\end{slide}
%-----------------------------------------------------------
\begin{slide}\ti{Object store}

Some key/value pairs can be automatically extracted:

\begin{itemize}
\item Author
\item Document type (picture, movie, book, etc)
\item Format (PDF, MPEG, etc)
\item Date of creation/modification
\end{itemize}

\vfill\end{slide}
%-----------------------------------------------------------
\begin{slide}\ti{Object store}

Other key/value pairs must be supplied manually:

\begin{itemize}
\item Language (natural language or programming language)
\item Project name
\end{itemize}

\vfill\end{slide}
%-----------------------------------------------------------
\begin{slide}\ti{Object store}

Directories can still exist.

But they will be objects just like any other object in the store.

\vfill\end{slide}
%-----------------------------------------------------------
\begin{slide}\ti{Object store}

The objects form a general graph.

Therefore a global garbage collector is needed.

Nowadays, there are very good concurrent garbage collectors.

Multi-core processors will help.

\vfill\end{slide}
%-----------------------------------------------------------
\begin{slide}\ti{Object store}

Most of the object store will fit in main memory.

In a typical system, most data is in the form of pictures, movies,
text, PDF-type documents, etc.

These documents do not contain pointers, so do not have to be scanned
by the garbage collector.

These documents will automatically migrate to secondary storage.

\vfill\end{slide}
%-----------------------------------------------------------
\begin{slide}\ti{Object store}

In the object store, each object is associated with an
\emph{access-control list} or ACL.

A user can always obtain any object from the object store, but may
have limited rights to operate on it.

\vfill\end{slide}
%-----------------------------------------------------------
\begin{slide}\ti{File}

In addition to Unix-style files, the system can handle other types of
objects:

\begin{itemize}
\item Typical data structures such as trees, lists, vectors, arrays,
  graphs.
\item Structured versions of typical ``business objects'' such as
  people, organizations, accounts.
\item Documents structured as chapters, sections, paragraphs.
\end{itemize}

\vfill\end{slide}
%-----------------------------------------------------------
\begin{slide}\ti{File}

Objects are referenced through \emph{capabilities}, i.e. pointers
augmented with access rights.

Accessing an object in the object store creates such a capability,
based on the access-control list in the store, and the user.

\vfill\end{slide}
%-----------------------------------------------------------
\begin{slide}\ti{Kernel}

Functions are called indirectly, so can be replaced arbitrarily.

\vfill\end{slide}
%-----------------------------------------------------------
\begin{slide}\ti{Monolithic applications}

  \begin{itemize}
  \item An application is just a collection of functions to be
    called.
  \item As with the kernel, functions are called indirectly, so can be
    replaced dynamically.
  \end{itemize}

\vfill\end{slide}
%-----------------------------------------------------------
\begin{slide}\ti{}

\vfill\end{slide}
%-----------------------------------------------------------
\begin{slide}\ti{}

\vfill\end{slide}
%-----------------------------------------------------------
\begin{slide}\ti{Technology that helps}

  \begin{itemize}
  \item Large address space.
  \item New permanent memory (3D Xpoint).
  \item Multi-core processors making concurrent garbage collection
    possible. 
  \item New concurrent garbage-collection algorithms.
  \end{itemize}

\vfill\end{slide}
%--------------------------------
\begin{slide}\ti{Current status}

Work in progress:

  \begin{itemize}
  \item Multi-user programming system to be used as the basis.
  \item Library for graphic user interfaces.
  \item Text editor.
  \end{itemize}

\vfill\end{slide}
%-----------------------------------------------------------
\begin{slide}\ti{Conclusions}

\vfill\end{slide}
%-----------------------------------------------------------
\begin{slide}\ti{Future work}

  \begin{itemize}
  \item Object-oriented device drivers inspired by I/O Kit by Apple.
  \item Implementation as a Unix process.
  \item Display server for graphics.
  \item Implementation as operating-system image (easy).
  \end{itemize}

\vfill\end{slide}
%% %-----------------------------------------------------------
%% \begin{slide}\ti{}

%% \vfill\end{slide}
%% %-----------------------------------------------------------
%% \begin{slide}\ti{}

%% \vfill\end{slide}
%% %-----------------------------------------------------------
%% \begin{slide}\ti{}

%% \vfill\end{slide}
%--------------------------------

\end{document}
 
