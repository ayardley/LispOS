\chapter{Checkpointing}
\label{chap-checkpointing}

The address of an object can be considered as consisting of two parts:
the \emph{page number} and the \emph{offset within the page}.  The
page number directly corresponds to the location on disk of the page.
However, when checkpointing is activated, the available disk memory is
divided into three parts, and the page number should be multiplied by
3 to get the first of three pages where the object might be located.%
\footnote{The price to pay for checkpointing is thus that disk memory
  will cost a factor 3 as much compared to the price when no
  checkpointing is used.}

Checkpointing is divided into \emph{cycles} delimited by
\emph{snapshots}.  At any point in time, two checkpointing cycles are
important.  The \emph{current} checkpointing cycle started at the
last snapshot and is still going on.  The \emph{previous}
checkpointing cycle is the one that ended at the last snapshot. 

A page can exist in one, two, or three \emph{versions}, located in
three different places on disk.  Version $0$ of the page is the oldest
version, and also the version that would be used when the system is
rebooted after a crash.  Version $0$ of the page always exists.
Version $1$ of the page corresponds to the contents of the page as it
was at the end of the \emph{previous} checkpoint cycle.  Version $1$
of the page exists if and only if the page was modified during the
previous checkpoint cycle.  Version $2$ of the page is the
\emph{current} version of the page.  Version $2$ of the page exists if
and only if the page has been modified since the beginning of the
\emph{current} checkpoint cycle.

A page can be associated with a \emph{frame}%
\footnote{A \emph{frame} is the main-memory instance of a page.}  An
attempt to access a page that is not associated with a frame results
in a \emph{page fault}.  At most one version of a particular page can
be associated with a frame, and then it is the version with the
highest number.  The frame associated with a page may or may not be
\emph{write protected}.  If it is write protected, any attempt to
modify the contents of the frame results in a \emph{write fault}.

%%  LocalWords:  checkpointing
