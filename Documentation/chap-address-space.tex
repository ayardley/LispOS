\chapter{Address space}

The system will have a single address space, as opposed to a separate
address space for each process.  We are targeting 64-bit
architectures, which allows us to consider all disk memory as being
part of the address space.  Basically, main memory is a cache for the
disk(s), and the virtual address of an object refers to its location
on disk (with some exceptions as discussed below).

Half the address space (say the upper half) contains thread-local
storage.  Each thread has a local heap and a stack.  For example, the
thread-local address space can be 16 or 32MiB of which the local heap
can be around 4MiB (smaller than the size of the cache).  Thread-local
storage must have associated physical disk memory. 

The other half of the address space (say the lower half) contains
mainly the global heap, aside from a small amount of space dedicated
to interrupt handlers and other architecture-imposed data structures.
The address space for the heap is divided in two halves.  Only one
half requires associated physical disk memory.  Thus, aside from the
architecture-imposed data structures, the two halves map to the same
disk space.  The reason for this organization is that it makes it easy
to implement the global garbage collector
\seechap{chap-garbage-collection}.  The method invented by Kermany et
al \cite{Kermany:2006:CCI:1133981.1134023} divides the heap address
space in two halves, but requires physical memory only for a little
more than half of it.

\refFig{fig-address-space} illustrates how the organization of the
address space might look on a computed equipped with the x86-64
processor.  This processor has a theoretical virtual address space of
$2^{64}$ bytes, but the real virtual address space might be limited
to $2^{46}$ bytes.  If so, then the middle of the theoretical address
space is not part of the real address space. 

\begin{figure}
\begin{center}
\inputfig{fig-address-space.pdf_t}
\end{center}
\caption{\label{fig-address-space}
Address space on x86-64.}
\end{figure}

As \refFig{fig-address-space} illustrates, the available disk space is
typically a small fraction of the possible virtual address space.  For
example, $2^{46}$ bytes is $64$ terabytes which is two orders of
magnitude larger than a typical hard disk.  In that case, the disk
space is divided (not necessarily in equal parts) between the global
heap and thread-local storage.  Furthermore, the size of each
thread-local storage can be configured according to whether a large
number of threads should be possible, or, on the contrary, each thread
should have a large stack.  

Main memory is not shown in \refFig{fig-address-space}, simply because
it just works as a demand-paged cache for the available disk space. 
