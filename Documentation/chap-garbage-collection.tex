\chapter{Garbage collection}

Contrary to traditional operating systems such as \unix{}, a Lisp
operating system will need a global \emph{garbage collector}.  There 
is a rich litterature (see e.g., \cite{Jones:2011:GCH:2025255}) on the
subject. 

Each thread has a local heap, roughly the size of the cache, say
around 4MiB.  The thread-local heap is managed entirely by the thread
itself, so that the garbage collector for it is executed by the thread
itself.  Experiments show that we will be able to run the thread-local
garbage collector in a few milliseconds, which is good enough for most
applications.  We will use a sliding garbage collector in order to
maintain allocation order.  This way, we have a precise measure of the
relative age of the objects, so that we can promote only the oldest
objects when required. 

In addition to the thread-local heaps, there is a global heap.  The
garbage collector for this heap will use the method invented by
Kermany et al \cite{Kermany:2006:CCI:1133981.1134023}.  It is also a
sliding garbage collector, except that it has some very good features,
including, parallel and concurrent execution.  These features allow us
to run the global garbage collector in multiple separate threads. 


