\chapter{Object store}
\label{chap-object-store}

The \emph{object store} is a system-wide database containing
any kind of objects.  Each object is a \emph{capability}. 

An object in the store can optionally be associated with a certain
number of \emph{attributes}.  An attribute is a \emph{pair} consisting
of the \emph{attribute name} and the \emph{attribute value}.  The
attribute name is a symbol in the \texttt{keyword} package.  The
attribute value can be any object. 

\begin{tabularx}{\linewidth}%
{|>{\setlength\hsize{.2\hsize}}X|%
  >{\setlength\hsize{.8\hsize}}X|}
\hline
Keyword & Possible-values\\
\hline\hline
\textbf{category} &

The nature of the object such
as \textbf{movie}, \textbf{music}, \textbf{article}, \textbf{book}, \textbf{user
manual}, \textbf{dictionary}, \textbf{course}, \textbf{lecture},
\textbf{recipe}, \textbf{program}, \textbf{bank statement},
\textbf{email}.  These would be chosen from an
editable set that is defined per user.\\
\hline
\textbf{name} &

A string that is displayed with the object, such as "A Dramatic
Turn of Events", "Three seasons", "Alternative energy".\\
\hline
\textbf{author} &

An object identifying a person, an organization, a company,
etc. \\
\hline
\textbf{genre} &

\textbf{progressive
metal}, \textbf{science}, \textbf{algorithms}, \textbf{garbage
collection}, \textbf{game}, \textbf{programming language
implementation}, \textbf{operating system}.  These would be
chosen from an editable set that is defined per user.\\
\hline
\textbf{format} &

This attribute can be used to identify the file type of documents such
as \textbf{PDF}, \textbf{ogg/vorbis}, \textbf{MPEG4} \textbf{PNG}, in
which case the attribute can be assigned automatically, but also to
identify the source format of files in a directory containing
things like articles or user manuals, for
example \textbf{LaTeX}, \textbf{Texinfo}, \textbf{HTML}.  These would
be chosen from an editable set that is defined per user. \\
\hline
\end{tabularx}

\begin{tabularx}{\linewidth}%
{|>{\setlength\hsize{.2\hsize}}X|%
  >{\setlength\hsize{.8\hsize}}X|}
\hline
Keyword & Possible-values\\
\hline\hline
\textbf{date of creation} &

A date interval.\\
\hline
\textbf{composer} &

An object representing a person.  On a compilation album there
can be more than one attribute of this kind. \\
\hline
\textbf{language} &

An object representing a natural language such
as \textbf{English}, \textbf{Vietnamese}, or a programming languages
such as \textbf{Lisp}, \textbf{Python}. These would
be chosen from an editable set that is defined per user.  If
appropriate, a document can have several of these attributes, for
instance if some program uses multiple programming languages, or
if a document is written using several languages, such as a
dictionary. \\
\hline
\textbf{duration} &

An object representing a duration. \\
\hline
\textbf{source control} &
 
\textbf{GIT}, \textbf{SVN}, \textbf{CVS}, \textbf{darcs}, etc.  These
would be chosen from an editable set that is defined per user.\\
\hline
\end{tabularx}

In a typical operating system installation, there are many fairly
large objects such as movies, music files, pictures, etc.  The amount
data associated with such an object that would be stored in the object
store is typically very small compared to the object itself.  Even a
fairly modest text file probably has $10^4 - 10^5$ characters in it,
whereas the meta-data probably takes up no more than $10^2 - 10^3$
bytes.  It is therefore likely that the entire object store will fit
in main memory.  Scanning the entire object store would then take at
most a few second of CPU time.  For better performance, one or more
\emph{indexes} could be created.  The objects could for instance be
divided by \emph{category}.

Searching the object store amounts to defining a \emph{filter},
i.e. a function that, given a set of keyword/value pairs returns
\emph{true} if and only if the corresponding object should be included
in the search result.  The result is returned to the user in the form
of a \emph{directory object} which is a list of \emph{object entries}
where each entry contains the object itself and the attributes of the
object from the store, if any.
